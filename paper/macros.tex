\usepackage{amsfonts}
\usepackage{amsmath}
\usepackage{amsthm}
\usepackage{amssymb}
\usepackage{dsfont}

\usepackage{xcolor}
%\usepackage[dvipsnames]{xcolor}
\usepackage{colortbl}
\usepackage{color}
\usepackage{graphicx}
\usepackage{multirow}
\usepackage{pifont}
\usepackage{comment}

\usepackage{mathtools}

% \usepackage[font=small]{caption}
% \usepackage[font=small]{subcaption}

% \usepackage[algo2e]{algorithm2e} 
\usepackage{verbatim}
\usepackage{xspace} %

\usepackage{enumerate}
\usepackage{enumitem}

\newcommand\tagthis{\addtocounter{equation}{1}\tag{\theequation}}

\newcommand{\dotprod}[2]{\left\langle #1,#2 \right\rangle}
\newcommand{\norms}[1]{\left\| #1 \right\|}
\newcommand{\expect}[1]{\mathbb{E}\left[ #1 \right]}


\definecolor{PineGreen}{HTML}{008B72}
\newcommand{\greencheck}{\color{PineGreen}\ding{51}}
\newcommand{\redx}{\color{red} \ding{55}}

\providecommand{\lin}[1]{\ensuremath{\left\langle #1 \right\rangle}}
\providecommand{\abs}[1]{\left\lvert#1\right\rvert}
\providecommand{\norm}[1]{\left\lVert#1\right\rVert}

\providecommand{\refLE}[1]{\ensuremath{\stackrel{(\ref{#1})}{\leq}}}
\providecommand{\refEQ}[1]{\ensuremath{\stackrel{(\ref{#1})}{=}}}
\providecommand{\refGE}[1]{\ensuremath{\stackrel{(\ref{#1})}{\geq}}}
\providecommand{\refID}[1]{\ensuremath{\stackrel{(\ref{#1})}{\equiv}}}

  \providecommand{\R}{\mathbb{R}} %
  \providecommand{\N}{\mathbb{N}} %
  
  \DeclareMathOperator{\E}{{\mathbb E}}
  \providecommand{\Eb}[1]{{\mathbb E}\left[#1\right] }       %
  \providecommand{\EE}[2]{{\mathbb E}_{#1}\left.#2\right. }  %
  \providecommand{\EEb}[2]{{\mathbb E}_{#1}\left[#2\right] } %
  \providecommand{\prob}[1]{{\rm Pr}\left[#1\right] } 
  \providecommand{\Prob}[2]{{\rm Pr}_{#1}\left[#2\right] } 
  \providecommand{\P}[1]{{\rm Pr}\left.#1\right. }
  \providecommand{\Pb}[1]{{\rm Pr}\left[#1\right] }
  \providecommand{\PP}[2]{{\rm Pr}_{#1}\left[#2\right] }
  \providecommand{\PPb}[2]{{\rm Pr}_{#1}\left[#2\right] }

  \DeclareMathOperator*{\argmin}{arg\,min}
  \DeclareMathOperator*{\argmax}{arg\,max}
  \DeclareMathOperator*{\supp}{supp}
  \DeclareMathOperator*{\diag}{diag}
  \DeclareMathOperator*{\Tr}{Tr}
  
  \DeclareMathOperator*{\Circ}{\bigcirc}
  
  \providecommand{\0}{\mathbf{0}}
  \providecommand{\1}{\mathbf{1}}
  \renewcommand{\aa}{\mathbf{a}}
  \providecommand{\bb}{\mathbf{b}}
  \providecommand{\cc}{\mathbf{c}}
  \providecommand{\dd}{\mathbf{d}}
  \providecommand{\ee}{\mathbf{e}}
  \providecommand{\ff}{\mathbf{f}}
  \let\ggg\gg
  \renewcommand{\gg}{\mathbf{g}}
  \providecommand{\hh}{\mathbf{h}}
  \providecommand{\ii}{\mathbf{i}}
  \providecommand{\jj}{\mathbf{j}}
  \providecommand{\kk}{\mathbf{k}}
  \let\lll\ll
  \renewcommand{\ll}{\mathbf{l}}
  \providecommand{\mm}{\mathbf{m}}
  \providecommand{\nn}{\mathbf{n}}
  \providecommand{\oo}{\mathbf{o}}
  \providecommand{\pp}{\mathbf{p}}
  \providecommand{\qq}{\mathbf{q}}
  \providecommand{\rr}{\mathbf{r}}
  \providecommand{\ss}{\mathbb{s}}
  \providecommand{\tt}{\mathbf{t}}
  \providecommand{\uu}{\mathbf{u}}
  \providecommand{\vv}{\mathbf{v}}
  \providecommand{\ww}{\mathbf{w}}
  \providecommand{\xx}{\mathbf{x}}
  \providecommand{\yy}{\mathbf{y}}
  \providecommand{\zz}{\mathbf{z}}
  
  \providecommand{\mA}{\mathbf{A}}
  \providecommand{\mB}{\mathbf{B}}
  \providecommand{\mC}{\mathbf{C}}
  \providecommand{\mD}{\mathbf{D}}
  \providecommand{\mE}{\mathbf{E}}
  \providecommand{\mF}{\mathbf{F}}
  \providecommand{\mG}{\mathbf{G}}
  \providecommand{\mH}{\mathbf{H}}
  \providecommand{\mI}{\mathbf{I}}
  \providecommand{\mJ}{\mathbf{J}}
  \providecommand{\mK}{\mathbf{K}}
  \providecommand{\mL}{\mathbf{L}}
  \providecommand{\mM}{\mathbf{M}}
  \providecommand{\mN}{\mathbf{N}}
  \providecommand{\mO}{\mathbf{O}}
  \providecommand{\mP}{\mathbf{P}}
  \providecommand{\mQ}{\mathbf{Q}}
  \providecommand{\mR}{\mathbf{R}}
  \providecommand{\mS}{\mathbf{S}}
  \providecommand{\mT}{\mathbf{T}}
  \providecommand{\mU}{\mathbf{U}}
  \providecommand{\mV}{\mathbf{V}}
  \providecommand{\mW}{\mathbf{W}}
  \providecommand{\mX}{\mathbf{X}}
  \providecommand{\mY}{\mathbf{Y}}
  \providecommand{\mZ}{\mathbf{Z}}
  \providecommand{\mLambda}{\mathbf{\Lambda}}
  
  \providecommand{\cA}{\mathcal{A}}
  \providecommand{\cB}{\mathcal{B}}
  \providecommand{\cC}{\mathcal{C}}
  \providecommand{\cD}{\mathcal{D}}
  \providecommand{\cE}{\mathcal{E}}
  \providecommand{\cF}{\mathcal{F}}
  \providecommand{\cG}{\mathcal{G}}
  \providecommand{\cH}{\mathcal{H}}
  \providecommand{\cJ}{\mathcal{J}}
  \providecommand{\cK}{\mathcal{K}}
  \providecommand{\cL}{\mathcal{L}}
  \providecommand{\cM}{\mathcal{M}}
  \providecommand{\cN}{\mathcal{N}}
  \providecommand{\cO}{\mathcal{O}}
  \providecommand{\cP}{\mathcal{P}}
  \providecommand{\cQ}{\mathcal{Q}}
  \providecommand{\cR}{\mathcal{R}}
  \providecommand{\cS}{\mathcal{S}}
  \providecommand{\cT}{\mathcal{T}}
  \providecommand{\cU}{\mathcal{U}}
  \providecommand{\cV}{\mathcal{V}}
  \providecommand{\cX}{\mathcal{X}}
  \providecommand{\cY}{\mathcal{Y}}
  \providecommand{\cW}{\mathcal{W}}
  \providecommand{\cZ}{\mathcal{Z}}
  
  
  \usepackage{bm}
  \newcommand{\bxi}{\boldsymbol{\xi}}

%\usepackage[textwidth=5cm]{todonotes}
% \usepackage[textsize=tiny]{todonotes}

% \providecommand{\mycomment}[3]{\todo[caption={},size=footnotesize,color=#1!20, inline]{\textbf{#2: }#3}}%
% \providecommand{\inlinecomment}[3]{%
%   {\color{#1}#2: #3}}%
% \newcommand\commenter[2]%
% {%
%   \expandafter\newcommand\csname i#1\endcsname[1]{\inlinecomment{#2}{#1}{##1}}
%   \expandafter\newcommand\csname #1\endcsname[1]{\mycomment{#2}{#1}{##1}}
% }


\newcommand{\circledOne}{\text{\ding{172}}}
\newcommand{\circledTwo}{\text{\ding{173}}}
\newcommand{\circledThree}{\text{\ding{174}}}
\newcommand{\circledFour}{\text{\ding{175}}}  
\newcommand{\circledFive}{\text{\ding{176}}} 
\newcommand{\circledSix}{\text{\ding{177}}}
  



\usepackage{url}
\def\UrlBreaks{\do\/\do-}
\def\ag#1{{\color{black}#1}} 

\newcommand{\AG}{\mathrm{AG}}

\newcommand{\prox}{\mathrm{prox}}
\newcommand{\proj}{\mathrm{proj}}
\newcommand{\range}{\mathrm{range}}
\newcommand{\Comp}{\mathrm{Comp}}
\newcommand{\Comm}{\mathrm{Comm}}
\newcommand{\Span}{\mathrm{Span}}
\newcommand{\Sp}{\mathrm{Sp}}
% \newcommand{\E}[1]{\mathbb{E}\left[#1\right]}
\newcommand{\Ek}[1]{\mathbb{E}_k\left[#1\right]}
\newcommand{\Ec}[2]{\mathbb{E}\left[#1\;\middle|\;#2\right]}
% \newcommand{\Prob}[1]{\mathrm{P}\left(#1\right)}
\newcommand{\cProb}[2]{\mathrm{P}\left(#1\;\middle|\;#2\right)}

\newcommand{\lmax}{\lambda_{\max}}
\newcommand{\lmin}{\lambda_{\min}}
\newcommand{\lminp}{\lambda_{\min}^+}
\newcommand{\sign}{\mathrm{sign}}
\newcommand{\g}{\nabla}
\newcommand{\ones}{\mathbf{1}}
\newcommand{\zeros}{\mathbf{0}}

\newcommand{\bg}{\mathrm{D}}

% \newcommand{\R}{\mathbb{R}}

\def\<#1,#2>{\langle #1,#2\rangle}

\newcommand{\Dotprod}[1]{\left\langle#1\right\rangle}


% \newcommand{\norm}[1]{\|#1\|}
\newcommand{\sqn}[1]{\norm{#1}^2}
\newcommand{\Norm}[1]{\left\|#1\right\|}
\newcommand{\sqN}[1]{\Norm{#1}^2}
\newcommand{\vect}[1]{\begin{bmatrix}#1\end{bmatrix}}
\newcommand{\infnorm}[1]{\norm{#1}_{\infty}}
\newcommand{\infNorm}[1]{\Norm{#1}_{\infty}}
% \newcommand{\diag}[1]{\mathrm{diag}\left(#1\right)}
\newcommand{\Diag}[1]{\mathrm{Diag}\left(#1\right)}

\newcommand{\sE}{\mathsf{E}}
% \newcommand{\cA}{\mathcal{A}}
% \newcommand{\cC}{\mathcal{C}}
% \newcommand{\cH}{\mathcal{H}}
% \newcommand{\cM}{\mathcal{M}}
% \newcommand{\cN}{\mathcal{N}}
% \newcommand{\cR}{\mathcal{R}}
% \newcommand{\cG}{\mathcal{G}}
% \newcommand{\cK}{\mathcal{K}}
% \newcommand{\cV}{\mathcal{V}}
% \newcommand{\cE}{\mathcal{E}}
% \newcommand{\cL}{\mathcal{L}}
% \newcommand{\cQ}{\mathcal{Q}}
% \newcommand{\cZ}{\mathcal{Z}}
% \newcommand{\cY}{\mathcal{Y}}
% \newcommand{\cX}{\mathcal{X}}
\newcommand{\sX}{{\mathsf E}}
% \newcommand{\cO}{\mathcal{O}}
\newcommand{\Lag}{\mathcal{L}}
% \newcommand{\mA}{\mathbf{A}}
% \newcommand{\mC}{\mathbf{C}}
% \newcommand{\mW}{\mathbf{W}}
% \newcommand{\mS}{\mathbf{S}}
% \newcommand{\mL}{\mathbf{L}}
% \newcommand{\mI}{\mathbf{I}}
% \newcommand{\mP}{\mathbf{P}}
\newcommand{\mSigma}{\mathbf{\Sigma}}
\newcommand{\sqNP}[1]{\Norm{#1}^2_\mP}
% \newcommand{\mM}{\mathbf{M}}
\newcommand{\mWp}{\mathbf{W}^{\dagger}}
\newcommand{\sqnw}[1]{\sqn{#1}_{\mWp}}
\newcommand{\eqdef}{\coloneqq}
% \DeclareMathOperator*{\argmin}{arg\,min}

\newcommand{\z}{\hat{z}}
\newcommand{\x}{\hat{x}}

\def\change#1{{\color{red}#1}}
\newcommand{\al}[1]{{\color{black}#1}} %Александр Лобанов 
\newcommand{\aw}[1]{{\color{black}#1}} %Андрей Веприков 

\usepackage[textwidth=5cm]{todonotes}
  
\providecommand{\mycomment}[3]{\todo[inline, caption={},size=footnotesize,color=#1!20]{\textbf{#2: }#3}}%
\newcommand\commenter[2]%
{%
  \expandafter\newcommand\csname #1\endcsname[1]{\mycomment{#2}{#1}{##1}}
}


%Комментарии  
\commenter{Aleksandr}{blue} % Александр Лобанов