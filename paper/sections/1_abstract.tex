\begin{center}
    \Large{\textbf{Аннотация}}
\end{center}

    В данной работе рассматривается проблема оптимизации <<черного ящика>>. В такой постановке задачи не имеется доступа к градиенту целевой функции, поэтому его необходимо как-то оценить. Предлагается новый способ аппроксимации градиента \texttt{JAGUAR}, который запоминает информацию из предыдущих итераций и требует $\mathcal{O}(1)$ обращений к оракулу. Эта аппроксимацию реализована для алгоритма Франка-Вульфа и для него доказана сходимость для выпуклой постановки задачи. Помимо детерминированной постановки рассматривается и стохастическая задача минимизации на множестве $Q$ с шумом в оракуле нулевого порядка, такая постановка довольно непопулярна в литературе, но я было доказано, что \texttt{JAGUAR} является робастной и в таком случае. Проведены эксперименты по сравнению оценщика градиента \texttt{JAGUAR} с уже известными в литературе и подтверждено его доминирование.